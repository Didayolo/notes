\documentclass{article}
\usepackage[utf8]{inputenc}
\usepackage{graphicx}
\usepackage{here}
\usepackage{amsmath}

\title{Notes TC4}
\author{Adrien Pavao}
\date{September 2017}

\begin{document}

\maketitle

\tableofcontents

\section{Inférence Bayesienne}

Différents niveaux d'inférence...

\subsection{Niveau 1 : Classification Bayesienne}

\begin{itemize}
\item Y : La classe à prédire (catégorielle)
\item $\vec{X}$ : Vecteur aléatoire, \( \vec{X} 
\begin{pmatrix} 
      x_1\\ 
      ...\\
      x_2 
\end{pmatrix} \)

\end{itemize}

On cherche à choisir y de façon à maximiser : 

\[ P(Y=y | \vec{X} = \vec{x}) = \frac{P(\vec{X} = \vec{x} | Y=y)P(Y=y)}{P(\vec{X} = \vec{x})} \]

Dans cette formule, on remarque des termes particuliers : 

\begin{itemize}

\item La \textbf{vraisemblance} : $P(\vec{X} = \vec{x} | Y = y)$.
\item L'\textbf{a priori} : $P(Y = y)$.
\item L'\textbf{évidence} : $P(\vec{X} = \vec{x})$.

\end{itemize}

La vraisemblance et l'a priori sont à estimer. On estime une ditribution sur X pour chaque classe y.
On peut donc faire l'hypothèse naïve suivante : 

\[ P(\vec{X}=\vec{x} | Y=y) = \Pi_{i=1}^{d} P(\vec{X}_i = \vec{x}_i | Y = y) \]

\subsubsection*{Estimer les paramètres}

Cas Bernouilli : $\Theta_{iy} = \frac{n(1, i, y)}{N(i, y)}$

$ n(1, i, y) =$ nombre de fois où $\vec{X}_i = 1$ dans la classe y.

Si $n(1, i, y) = 0$ alors $\Theta_{iy} = 0$
Donc $P(\vec{X} = \vec{x} | Y = y) = 0$, ce qui est mauvais. On estime $\Theta$ sur les données et on vient à la conclusion qu'un evenement est impossible sous pretexte qu'on ne l'a jamais observé. Il faut éviter ce problème.

Ce type d'estimation est appelée une estimation MLE : Maximum Likelihood Estimate. Il s'agit de l'interprétation \textbf{fréquentiste} des données.

Autrement dit, on cherche les paramètres $\Theta_{iy}$ qui maximisent $P(D | \Theta_{iy})$. (D la réalisation des données ..)

\subsection{Niveau 2 : Inférence Bayesienne des paramètres}

On cherche $P(X_i | Y)$ -> $P(X_i | Y_i \Theta_{iy})$. L'apprentissage revient à l'estimation d'une distribution sur les paramètres.

Estimer $ P(\Theta_{iy} | D) $.

\[ P(\Theta_{iy} | D) = \frac{P(D | \Theta_{iy})P(\Theta_{iy})}{P(D)} \]

\subsubsection{A priori sur les paramètres}

Cas Bernouilli : $\Theta_{iy} \in [0, 1]$, continu. Donc $P(\Theta_{iy})$ \_ une loi continue de support $[0, 1]$.
Le choix : Loi Beta.

\[ P(\Theta_{iy}; \alpha_0, \alpha_1) = \frac{\Gamma (\alpha_0 + \alpha_1)}{\Gamma (\alpha_0) \Gamma (\alpha_1)} \Theta_{iy}^{\alpha_1 - 1} (1 - \Theta_{iy})^{\alpha_0 - 1} \]

(Dénominateur et game -> Normalisation)

$\alpha_0$ et $\alpha_1$ sont les paramètres de la loi Beta. On a $\alpha_0, \alpha_1 > 0, \in R$ (R reel, D majuscule ...)

\begin{itemize}
\item \textbf{Fonction de densité symétrique : }
$\alpha_0 = \alpha_1$ et $\alpha_0, \alpha_1 > 1$.

Graphe 1

\item \textbf{A priori non-informatif : }

$\alpha_0 = \alpha_1 = 1$.

Graphe 2

\item \textbf{A priori parcimonieux (sparse) : }

$\alpha_0, \alpha_1 < 1$

Graphe 3

\end{itemize}

\subsubsection{A posteriori sur les paramètres}

\[ P(\Theta_{iy} | D) \propto P(D | \Theta_{iy}) P(\Theta_{iy}; \alpha_1, \alpha_0) \] (vraimsemblance et a priori).
\[ P(\Theta_{iy} | D) \propto \Theta_{iy}^{N_1 + \alpha_1 - 1} (1 - \Theta_{iy})^{N_0 + \alpha_0 - 1} \]

$\propto$ signifie "proportionnel à".

\begin{itemize}
\item $N_0$ : Nombre de $x_i$ à 0 dans D.
\item $N_1$ : Nombre de $x_i$ à 1 dans D.
\end{itemize}

(defition importante)
La loi a posteriori est comme la loi a priori, une loi Beta. La loi Beta est l'a priori \textbf{conjugué} de Bernouilli (conjugated prior).

\subsubsection{Retour à la classification}

\begin{enumerate}

\item \textbf{Maximum a Posteriori des Paramètres (MAP)}

$ \Theta_{iy} = argmax P(\Theta_{iy} | D) $ ( chapeau sur le theta ! )
$ \Theta_{iy} = \frac{N_1 + \alpha_1 - 1}{N_1 + N_0 + \alpha_1 + \alpha_0 - 2} $

$\alpha_1$ et $\alpha_0$ agissent comme des "pseudo-comptes". Lissage (smoothing) de distibution.
$\Theta_{iy} != 0$
Si $N_1, N_0 >> \alpha_1, \alpha_0$ alors l'a priori est négligeable.

-> Régularisation, éviter le sur-apprentissage.

\item \textbf{Loi prédictive (inférence Bayesienne 3)}

$P(X_i = x_i | Y = y; \Theta_{iy})$ avec $\Theta_{iy}$ estimés à partir des données (MAP).

Le paramètre n'existe pas et ne doit donc pas apparaitre dans la prédiction. La vraie prédiction :

$P(X_i = x_i | D) = integrale01 P(X_i = x_i; \Theta_{iy} | D) d \Theta_{iy}$, en marginalisant les paramètres.

$P(X_i; \Theta_{iy} | D) = P(X_i | \Theta_{iy}; D) P(\Theta_{iy} | D)$ (vraisemblance et a priori).

$ P(X_i = x_i | D) = \frac{N_1 + \alpha_1}{N_1 + N_0 + \alpha_1 + \alpha_0}$, $\forall \alpha_1$ et $\alpha_0 > 0$.

\end{enumerate}

\section{Modèles de mélange (G.M.M.)}

\subsection{Introduction}

Un large champ d'applications :

\begin{itemize}

\item \textbf{Clustering :} Apprentissage non supervisé. Par exemple, l'algorithme des K-means.

\[ D = {(x_n)^N}_{n=1}  \]

On fixe K, un nombre de clusters.

\item \textbf{Estimation de distribution.}

Exemple : La classification (d'image).

Graphe 1.

\begin{itemize}
\item Augmenter la capacité du modèle.
\item Augmenter le nombre de paramètres.
\end{itemize}

\item \textbf{Mélange de Gaussienne (G.M.M.)}

K : Le nombre de Gausiennes / clusters.

\[ P(\vec{x_n} | \Theta) = \sum_{k=1}^k \pi_k N(\vec{u_k}, \Sigma_k) \]

\begin{itemize}

\item Les paramètres $\Theta$ : $(\pi_k, \vec{u_k}, \sum_k)_{k=1}^K$

\item $\pi_k$ est le poids du mélange.

\item $N(\vec{u_k}, \Sigma_k)$ est la loi gaussienne.

\end{itemize}

\end{itemize}

L'objectif de l'apprentissage est d'estimer les paramètres du mélange permettant de :

\begin{itemize}
\item Maximiser $ \Pi_{n=1}^N  P(\vec{X} = \vec{x} | \Theta)$

\item Maximiser $log(\Pi_{n=1}^N P(\vec{X} = \vec{x_n} | \Theta))$ (on retrouve la probabilité vue plus haut).

\end{itemize}

\subsection{Algorithme E.M.}

\begin{itemize}

\item Algorithme itératif qui cherche à maximiser : 

\[ log(P(\vec{X} = \vec{x_n} | \Theta))  \]

\item Introduire des variables \textbf{latentes} (cachées) : 

\begin{itemize}

\item Pour chaque $\vec{x} -> \vec{Z}$ (one-hot vecteur)
\item $\vec{Z} = (0, 0, ..., 1, 0, 0) -> Z_k = 1 <=> \vec{x} \in cluster k$
\item $\vec{Z}$ : 
\begin{itemize}
\item Pseudo-affectation
\item Un vecteur latent
\item Inconnu => $\vec{Z}$ un vecteur aléatoire
\item Affectation "soft" : Un point peut appartenir à tous les clusters.
\end{itemize}
\end{itemize}
\end{itemize}

Résumé du programme : 

Introduction $\vec{Z}$ associé à $\vec{X}$. Si on souhaite maximiser : 
\[ P(X | \Theta) = \sum_Z P(\vec{X}, \vec{Z} | \Theta) \]
\[ P(X | \Theta) = \sum_Z P(\vec{X} | \vec{Z}, \Theta) P(\vec{Z} | \Theta) \]

On note que $P(X | Z, \Theta)$ est la loi normale $N(\vec{u_k}, \Sigma_k)$ et que $P(\vec{Z} | \Theta)$ est $\pi_k$.

Si $\vec{Z_k} = (0, ..., 1, 0) rang k$
\begin{itemize}
\item $(\vec{X}, \vec{Z})$ : Données complètes.
\item $(\vec{X})$ : Données incomplètes.
\end{itemize}

\textbf{Etape E(xpection) : }
\begin{itemize}
\item Connaitre $\vec{Z}$ à $\Theta$ \textbf{fixé}.
\item Calcul la probabilité d'affectation : $P(\vec{Z} |, \vec{X}, \Theta)$
\end{itemize}

\textbf{Etape M(aximization) :} Les données sont incomplètes. On calcule $\Theta$ et on "fixe" $\vec{Z}$.

\subsection{Optimisation variationnelle}

Après l'introduction de $\vec{Z}$, on introduit une distribution auxiliaire sur $\vec{Z}$, notée $q(\vec{Z})$. On souhaite maximiser selon $\Theta$ : 

\[ log(P(X | \Theta) = \sum_{\vec{Z}} q(\vec{Z} log(\frac{P(\vec{X}, \vec{Z} | \Theta)}{q(\vec{Z})} ) - \sum_{\vec{Z}} q(\vec{Z} log(\frac{P(\vec{Z} | \vec{X}, \Theta)}{q(\vec{Z})} ) \]

\[ log(P(X | \Theta)) = log(P(X, Z | \Theta)) - log(P(Z | X, \Theta)) \]

Rappel : $P(X | \Theta) = \frac{P(X, Z | \Theta)}{P(Z | X, \Theta)}$

C'est-à-dire : Le second terme : 

\[ - \sum_{\vec{Z}} q(\vec{Z} log(\frac{P(\vec{Z} | \vec{X}, \Theta)}{q(\vec{Z})} ) = E_{\vec{Z}vq(\vec{Z})} [log(\frac{P(\vec{Z}, \vec{X} | \Theta}{q(\vec{Z})})]  \]

Divergence de Kullback-Leibler (DKL).

\[ DKL(q(\vec{Z}) || P(\vec{Z} | \vec{X}, \Theta)) \]

De chaque côté du "$||$" on a deux distributions sur $\vec{Z}$.

Divergence $\ne$ distance (asymétrique). (faire une phrase...)
\begin{itemize}
\item $DKL(q, P) = 0$ ssi $q = P$
\item $DKL(q, P) \geq 0$ 
\end{itemize}

Le premier terme : $ E_{\vec{Z} v q(\vec{Z})} [log(\frac{P(\vec{Z}, \vec{X} | \Theta}{q(\vec{Z})}))] $ est nommé ELBO (Evidence Lower Bound).

\[ log(P(\vec{X} | \Theta)) = L(\Theta, q) + DKL(q(\vec{Z}) || P(\vec{Z} | \vec{X}, \Theta))  \]

On a $L(\Theta, q)$ une borne inférieure (ELBO). On fait une optimisation par borne inférieure : on maximise la fonction en maximisant sa borne inférieure. Il s'agit d'une maximisation "indirecte".

\textbf{Etape E :} 
\begin{itemize}
\item Les paramètres sont fixés : $\Theta = \Theta^{old}$
\item Maximiser $L(\Theta^{old}, q)$

\[ L(\Theta^{old}, q) = - DKL(q(\vec{Z}), P(\vec{Z} | \vec{X}, \Theta^{old})) + log (P(\vec{X} | \Theta^{old})) \]
\[ q(\vec{Z}) = P(\vec{Z} | \vec{X}, \Theta^{old})  \]

\textbf{Etape M :} Maximiser L selon $\Theta$ avec q fixé.

ILLUSTRATION..
\end{itemize}
\end{document}


