\documentclass{article}
\usepackage[utf8]{inputenc}

\title{Notes TC4}
\author{Adrien Pavao}
\date{September 2017}

\begin{document}

\maketitle

\tableofcontents

\section{Définitions et formules}

\subsection{Notions générales}

\begin{itemize}
\item \textbf{Variable aléatoire :} Une fonction définie depuis l'ensemble des résultats possibles d'une expérience aléatoire, dont on doit pouvoir déterminer la probabilité qu'elle prenne une valeur donnée ou un ensemble donné de valeurs. 

Discrete, continue

Fonction de masse
Fonction de densité

\item \textbf{Réalisations :} Les réalisations d'une variable aléatoire sont les résultats des valeurs choisies au hasard en fonction de la loi de probabilité de la variable. On les appelle également les variations aléatoires.

\item \textbf{Distribution (loi de probabilité) :} Le concept de loi de probabilité se formalise mathématiquement à l'aide de la théorie de la mesure : une loi de probabilité est une mesure, souvent vue comme la loi décrivant le comportement d'une variable aléatoire, discrète ou continue. Une mesure est une loi de probabilité si sa masse totale vaut 1. L'étude d'une variable aléatoire suivant une loi de probabilité discrète fait apparaître des calculs de sommes et de séries, alors que si sa loi est absolument continue, l'étude de la variable aléatoire fait apparaître des calculs d'intégrales.

\item \textbf{Inférence :} Trouver la valeur des v.a. à partir d'autres qui sont connues.

\item \textbf{Estimation :} Retrouver les paramètres d'une distribution à partir de l'observation d'un ensemble de réalisations de celle-ci.

\item \textbf{Sampling :} Générer des données.

\end{itemize}

\subsection{Différentes probabilités liées}

\begin{itemize}

\item \textbf{Probabilité jointe :} Probabilité d'une configuration donnée. 
\item \textbf{Probabilité conditionnelle :} Game sachant game (formule) 
\item \textbf{Probabilité marginale :} Game (formule)

\end{itemize}

La probabilité jointe inclus les deux autres. (formule)

\subsection{Indépendance statistique}

Covariance tanani

\begin{itemize}

\item \textbf{Variables indépendantes :} jointe a * b
\item \textbf{Variables conditionnellement indépedantes :} (formule)

\item \textbf{Indépendantes et identiquement distribuées (i.i.d) :} game

\end{itemize}

\subsection{Bayes}

\begin{itemize}

\item \textbf{Théorème de Bayes :} (Formulus)

\end{itemize}


\end{document}

